\documentclass[t]{beamer}

\usepackage[utf8]{inputenc}
\usepackage[T1]{fontenc}
\usepackage{helvet}
\usepackage[english]{babel}

\usepackage[
backend=biber,
style=nejm,
sorting=ynt
]{biblatex}

\addbibresource{bibliography.bib}
 
\usetheme{LUH}



\title{Some Presentation Title That's Long Enough To Need Two Lines}
\date[\today]{Converence XYZ, \today}
\author[Insert author]{Author DEF}

\setbeamertemplate{logonortheastp1}{\begin{picture}(0,0)(0,0)
\put(-0,-30){\llap{\includegraphics[width=.95\textwidth]{hautp1.png}}}\end{picture}}
\setbeamertemplate{logonorthwestp1}{}
\setbeamertemplate{logosoutheastp1}{\begin{picture}(0,0)(20,2)
\put(0,20){\llap{
\includegraphics[height=1cm]{logoAPT.png}%
\includegraphics[height=1cm]{logoINRA.png}}}~
\end{picture}
}
\setbeamertemplate{logosouthwestp1}{\begin{picture}(0,0)(-5,-1)
\put(50,0){\llap{\includegraphics[height=19mm]{coinbasgauche.jpg}}}
\put(12,53){\includegraphics[height=.93\textheight]{traitgauche.png}}
\put(45,6){
\includegraphics[width=.88\textwidth]{traitbas.png}}
\put(20,32){\color{red}Ingénierie Procédés Aliments}
\put(20,22){\color{red}\textit{(Food \& Process Engineering)}}
\end{picture}
}
% \titleimage{\includegraphics[width=.7\paperwidth]{luh_title_image}}
\setbeamertemplate{logonortheast}{\begin{picture}(0,0)(0,0)
\put(0,-20){\llap{\includegraphics[width=.94\textwidth]{haut.png}}}
\end{picture}}
\setbeamertemplate{logosouthwest}{\begin{picture}(0,0)(-5,-1)
\put(50,0){\llap{\includegraphics[height=19mm]{coinbasgauche.jpg}}}
\put(45,6){
\includegraphics[width=.88\textwidth]{traitbas.png}}
\put(-5,53){\rotatebox{90}{\rlap{
\includegraphics[height=5mm]{logoAPT.png}%
\includegraphics[height=5mm]{logoINRA.png}}}}
\put(12,53){\includegraphics[height=.93\textheight]{traitgauche.png}}
\end{picture}
}
% \setbeamertemplate{logosoutheast}{\includegraphics[height=\LUHLogoHeight]{traitbas.png}}

\AtBeginSection[]% merci http://jeromyanglim.tumblr.com/post/33559584570/how-to-show-active-section-in-table-of-contents-in
{
   \begin{frame}
       \tableofcontents[currentsection]
   \end{frame}
}

% \usepackage[para]{footepackage{parnotes}% marche pas dans beamer.
\usepackage{parnotes}% pour \parnote https://tex.stackexchange.com/questions/271809/footnotes-in-a-single-paragraph-in-beamer

\DeclareSourcemap{
    \maps[datatype=bibtex,overwrite=true]{
        \map[refsection=1]{% with refsection=<n> the map is valid only for the refsection with <n> number
            \step[fieldsource=title,
            match=\regexp{The\sMathematics\sof\sDiffusion},
            replace={MathDiffus}]
            \step[fieldsource=title,
            match=\regexp{.*\s.*},
            replace={}]
            \step[fieldsource=journal,
            match=\regexp{Drying\sTechnology},
            replace={DryTech}]
            \step[fieldsource=journal,
            match=\regexp{Journal\sof\sFood\sEngineering},
            replace={JFoodEng}]
            \step[fieldsource=journal,
            match=\regexp{Industrial\s.*Engineering\sChemistry\sResearch},
            replace={IndEngChemRes}]
            \step[fieldsource=number,
            match=\regexp{.*},
            replace={}] 
            \step[fieldsource=pages,
            match=\regexp{.*},
            replace={}] 
            \step[fieldsource=volume,
            match=\regexp{.*},
            replace={}] 
            \step[fieldsource=publisher,
            match=\regexp{.*},
            replace={}] 
            \step[fieldsource=doi,
            match=\regexp{.*},
            replace={}] 
            \step[fieldsource=author,
            match=\regexp{[, ].*},
            replace={}]
            \step[fieldsource=year, % https://tex.stackexchange.com/questions/142837/declaresourcemap-in-biblatex
            match=\regexp{19([3-9][0-9])},
            replace={'$1}]
            \step[fieldsource=year, % https://tex.stackexchange.com/questions/142837/declaresourcemap-in-biblatex
            match=\regexp{20([0-2][0-9])},
            replace={'$1}]
        }                                     
    }
}
\global\def\bibliofinale{}
% \renewcommand{\parnoteintercmd}{\penalty-1000\hskip 1em plus 10em minus 0.2em}
\makeatletter
\newcommand{\parcite}[2][]{\parnote{#1\fullcite{#2}}
    \unless\ifx\bibliofinale\@empty\g@addto@macro\bibliofinale{\parnoteintercmd}\fi
    % Redefine \@currentlabel to the parnote label, so \label works
    \g@addto@macro\bibliofinale{\phantomsection\def\@currentlabel{#1}}%
    \g@addto@macro\bibliofinale{\cite{#2}}%
}
\makeatother
\let\tobefullcite=\fullcite

\begin{document}

\begin{frame}
\titlepage
\end{frame}

\begin{refsection}
\section{premiere section} 


\begin{frame}{Some frame title}
Some text\parcite{Romdhana19DryTech} \parcite{Crank80theMathematics} \parcite{Vitrac06identification} \parcite{Romdhana16JFoodEng}
\begin{itemize}
  \item Hello,
    \begin{itemize}
      \item cruel
        \begin{itemize}
          \item world!
        \end{itemize}
    \end{itemize}
    \item Example
\end{itemize}


\begin{enumerate}
  \item Hello,
    \begin{enumerate}
      \item cruel
        \begin{enumerate}
          \item world!
        \end{enumerate}
    \end{enumerate}
  \item Example
\end{enumerate}

\end{frame}

\section{seconde section}


\begin{frame}{Some other frame}
\framesubtitle{With a subtitle}
\begin{block}{Block Header}
And some content.\parcite{Romdhana19DryTech}

And some more content.

And even more content.

And some more content.

And even more content.

And some more content.

And even more content.

And some more content.

And even more content.

And some more content.

And even more content.

And some more content.

And even more content.

And some more content.

And even more content.

And some more content.

And even more content.

And some more content.

And even more content.

And some more content.

And even more content.

And some more content.

And even more content.

And some more content.

And even more content.

And some more content.

And even more content.

And some more content.

And even more content.

And some more content.

And even more content.

And some more content.

And even more content.

And some more content.

And even more content.

And some more content.

And even more content.

And some more content.

And even more content.

And some more content.

And even more content.


And some more content.

And even more content.
\end{block}
\end{frame}

\end{refsection}

\begin{frame}
\setbox0=\hbox{\bibliofinale}
\printbibliography
\end{frame}
\end{document}
